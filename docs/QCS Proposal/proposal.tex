\documentclass[12pt]{article}

% Any percent sign marks a comment to the end of the line

% Every latex document starts with a documentclass declaration like this
% The option dvips allows for graphics, 12pt is the font size, and article
%   is the style

\usepackage[pdftex]{graphicx}
\usepackage{url}
\usepackage[utf8]{inputenc}
\input{Qcircuit}

% These are additional packages for "pdflatex", graphics, and to include
% hyperlinks inside a document.

\setlength{\oddsidemargin}{0.25in}
\setlength{\textwidth}{6.5in}
\setlength{\topmargin}{0in}
\setlength{\textheight}{8.5in}

% These force using more of the margins that is the default style

\begin{document}

% Everything after this becomes content
% Replace the text between curly brackets with your own

\title{Quantum Circuit Simulator for \\
Mobile Devices Running Android OS}
\author{Mert Can ÇIKLA}
\date{\today}

% You can leave out "date" and it will be added automatically for today
% You can change the "\today" date to any text you like
 
 



\maketitle

% This command causes the title to be created in the document

\section{Introduction}

% An article style is separated into sections and subsections with 
%   markup such as this.  Use \section*{Principles} for unnumbered sections.
	Quantum computers use the parallel computation power that is hidden
within objects that are governed by quantum physics rather than classic physics such as a photon.
Due to its power to solve primitive problems such as Prime Factorization significantly faster 
than current silicon based computers. Quantum computation is getting more and more attention 
every day and its field of research is expanding exponentially.\\
 
 With these in mind the proposed
plan is to build an application called \textbf{Quantum Circuit Simulator}
that runs on the widely popular Android mobile operating system. 

\subsection{Target Devices}
Mobile devices that are running Android 3.0 or above such as tablets or cellphones with large screens.
\subsection{Target Users}
Students studying Mathematics, Physics or Computer Engineering are the main 
target as users of this application.
\section{Application Description}

On top of the application is the toolbar where the basic gates such as Hadamard,Pauli-X,
 Pauli-Y, Pauli-Z, Swap and Cnot are. Next to the gates is there is a spinner that lets 
 the user select how many qubits he/she would like to work on. The qubits will start with 
 state   $\vert 0  \rangle$  and on click will swap to  $\vert 1 \rangle$ and go back to 
 state  $\vert 0 \rangle$  with the second click. The gates on the toolbar can be dragged 
 and dropped to the main area where the circuit will be defined. Gates can be removed from 
 the circuit by dragging them back to the toolbar. By clicking a button on the toolbar the 
 calculation will start and multiply the qubit which is represented by a 2d vector with 
 complex entries. Gates are represented with either 2x2 or 4x4 matrices which are also comprised of
 complex entries. The result of this multiplication is the 2d output vector. An example of this operation is
 shown in Figure 1.





\section{Objectives}
\begin{figure}[tbp]



\Qcircuit @C=0.6em @R=0.6em {
&&&&&&&&&&&&&&&&&&&&&&&&&&\lstick{\ket{0}} & \gate{H} & \qw &
\rstick{\frac{1}{\sqrt{2}}  \ket{0}+\frac{1}{\sqrt{2}}\ket{1}}}

\caption{blabla}
\end{figure}
\subsection{Ease of Use}

Application is aimed to be fairly easy to use with a clean and simple GUI. In order to achieve this,
drag/drop framework within Android API 11 will be used. 

\subsection{Functionality}

Application is aimed to be able to simulate behaviour of any quantum 
gate or circuit that acts on up to 8 qubits or possibly less depending on screen resolution
of the device running.

\section{Challenges}




\end{document}
